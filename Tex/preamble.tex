\usepackage{amsmath}
\usepackage{mathptmx}
\usepackage{ctex}
\usepackage{minted}
\usepackage{tcolorbox}
\usepackage{epigraph}
\usepackage{caption}
\usepackage{graphicx}
\setkeys{Gin}{width=0.75\textwidth}
\usepackage{listings}
\newtheorem{theorem}{Theorem}


\setminted[python]{bgcolor=orange!5, linenos, breakanywhere=true}
%定义新的minted命令
\newminted[pyc]{python}{}


\title{\textbf{Python for Data Analysis, 3rd edition} \\
\large Data Wrangling with pandas, NumPy, and Jupyter
}
	
\author{Stephen CUI\thanks{cuixuanStephen@gmail.com}}
\date{January 4, 2022}
\newcommand\explain[2]{
\begin{tcolorbox}[breakable,colback=orange!5,colframe=orange!95!black,title=\textbf{#1}]
#2
\end{tcolorbox}
}

\newcommand\tips[1]{
	\begin{tcolorbox}[breakable, colback=green!7, colframe=green!70!black,title=Suggestions]
		#1
	\end{tcolorbox}
}

\newcommand\notes[1]{
	\begin{tcolorbox}[breakable, colback=blue!7, colframe=blue!70!black,title=Notes]
		#1
	\end{tcolorbox}
}

\newcommand\warning[1]{
	\begin{tcolorbox}[breakable, colback=red!7, colframe=red!70!black,title=Warnings]
		#1
	\end{tcolorbox}
}
% \newcommand\tips[1]{\textcolor{green!70!black}{\textbf{Tips: } #1}}
% \newcommand\notes[1]{\textcolor{blue!70!black}{\textbf{Notes: } #1}}
% \newcommand\warning[1]{\textcolor{orange!90!black}{\textbf{Warnings: } #1}}
\newcommand\figures[1]{
\begin{figure}
\centering
\includegraphics{img/codes/#1.png}
\caption{#1}
\label{#1}
\end{figure}
}


\tcbuselibrary{minted, skins, breakable}
\newtcblisting[auto counter, number within =section]{py}[1]{listing engine=minted,
						 minted style=colorful,
						 minted language=python,
						 minted options={breaklines,autogobble,linenos,numbersep=3mm},
						 colback=orange!5!white,colframe=orange!70!black,listing only, left=5mm,enhanced,
						 title=Examples~\thetcbcounter~#1,
						 breakable,
						 enhanced
%						 overlay={
%						 	\begin{tcbclipinterior}
%						 		\fill[red!30!white] (frame.south west)
%						 		rectangle ([xshift=5mm]frame.north west);
%							\end{tcbclipinterior}}
							}
							
\usepackage{geometry}
\geometry{left=2.45cm,right=2.45cm, top=2.77cm, bottom=2.77cm}

\usepackage{hyperref}
% 将引用的chapter改写为Chapter
\usepackage[english]{babel}
\addto\extrasenglish{
	\def\chapterautorefname{Chapter}
}
\hypersetup{
	colorlinks=true,
	linkcolor=orange!95!black,
	filecolor=blue,      
	urlcolor=red!75!black,
	citecolor=green,
}


